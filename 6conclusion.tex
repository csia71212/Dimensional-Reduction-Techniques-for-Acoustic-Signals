\section{Summary}


Throughout this dissertation we have introduced the fundamentals of dimensional reduction and demonstrated practical application via R language on acoustic signals.

We have thoroughly introduced two major dimensional reduction techniques: principal component analysis (PCA) and independent component analysis (ICA). We have narrated the general characteristics, mathematical definitions, and applicational process of both techniques with real-life examples and precise logical reasoning.

We have introduced and shown several pre-processing of data, such as interpolation, centering, whitening, etc... , by actual demonstration on 219 acoustic signals of Romance languages via R language. We have also narrated and demonstrated the transformation of the processed data into spectrum and have produced visual representation as spectrograms.

Lastly, we have successfully implemented the PCA and ICA techniques on the processed data and deliberately analyzed the results with various visual representations. 

We have applied the PCA method on interpolated data, spectrum data, and on data sorted by different types of languages. We have carefully executed each process with deliberate coding and have successfully narrated the numeric results of each progress.

We have applied the ICA method on interpolated data on various input conditions, and have well-plotted the visual representations of the results. We have affirmed our initial assumption on the effectiveness of components throughout the five computation process.

By analyzing the numeric and visual results from both methods, we can conclude that the PCA method is more suitable and efficient when performing dimensional reduction on acoustic signal data, which has a substantially high volume of variables (or in other words dimensions). We approached a result of 84 percent reduction in dimension with 80 percent of total variance for spectrum dataframe, and 64 percent reduction in dimension with 90 percent of total variance for interpolated dataframe.

In conclusion, the following dissertation well-explained the fundamentals of dimensional reduction and thoroughly demonstrated the application of the dimensional techniques on acoustic signals via R language.



