\section{Introduction}
In data analysis, dimensional reduction is the process of reducing the number of independent variables of a data set in order to well-visualise the data or to improve the efficiency of the analysis \cite{Malik}. 

There exists various types of dimensional reduction techniques with different characteristics and method of approach but under the same purpose: to reduce the dimension of the data.

Reducing the dimension of a data is useful for number of reasons. 

First, it improves efficiency. When processing or analyzing a data set, the number of variables in the data set may be significantly large, making the computational process highly time-consuming. Dimensional reduction techniques can aid to reduce the number of variables (or dimensions) in the data set, which makes the analysis much more efficient.

Second, it provides better visualization. Data are best visually represented in 2 or 3 dimension, where the human eye and brain are
most accustomed to. Lowering the dimension of a data set into 2 or 3 dimension and presenting the data into a visual model will make it easier to explore and analyze the data.

Overall, dimensional reduction is a useful tool for simplifying high-dimension data and makes it easier to analyze it.

The aim of the project is to explore some of the most well-known dimensional reduction techniques, study their mathematical definition, apply them to various acoustic signals on Romance languages via R language, and analyze the results in various numerically and visually represented models \cite{R}. 219 acoustic signals of various Romance languages (Portuguese, Italian, Iberian Spanish, French, American Spanish), each comprising between 5 to 15 seconds, were given prior to this project as a sample data set \cite{Pigoli}.

