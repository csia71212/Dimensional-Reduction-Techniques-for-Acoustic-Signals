\section*{Appendix}


In this section we will show the actual code lines used for this dissertation

This is the code used for interpolating the original Romance Digit data file into a dataframe and plotting relevant figures.
\begin{verbatim}
########## Interpolation ############


itpData <- matrix(0, ncol = 5000, nrow = 1)

for (i in 1:length(Data)){
  ytemp <- data.frame(Data[i])
  y <- c()
  
  for (j in 1:nrow(ytemp)){
    y <- append(y, ytemp[j,])
  }
  
  x <- c(1:nrow(ytemp))
  
  ##plot(x, y)
  
  xoutval = seq(1,nrow(ytemp),len=5000)
  templist <- approx(x, y, xout=xoutval, method='linear')
  
  ##plot(templist)
  
  tempylist <- unlist(templist[2])
  itpData <- rbind(itpData, tempylist)
}

finalDF <- data.frame(itpData)
finalDF <- finalDF[-1, ]

for (i in 1:nrow(finalDF)){
  rownames(finalDF)[i] <- names(Data)[i]
}

saveDF <- finalDF

rm(itpData)
rm(my_pca)
rm(templist)
rm(ytemp)
\end{verbatim}

This is the code used for transforming the interpolated dataframe into a spectrum and plotting relevant figures such as spectrograms
\begin{verbatim}
################## Spectrogram ################


library(signal, warn.conflicts = F, quietly = T) # signal processing functions
library(oce, warn.conflicts = F, quietly = T) # image plotting functions and nice color maps
library(tuneR, warn.conflicts = F, quietly = T) 

specData <- matrix(0, ncol = 19456, nrow = 1)

nfft=1024
window=256
overlap=128
dur=5
fs=1000

for (i in 1:length(Data)){
  snd = unlist(finalDF[i,])
  mean(snd)
  snd = snd - mean(snd)
  
  setwd("C:/Users/csia7/Desktop/OneDrive/OneDrive_2022-11-08/6CCM345A Third Year Project/R/R images/Spectrum(new)/Romance/Wave")
  pdf(paste(gsub("c", "", rownames(finalDF)[i]), "pdf", sep=""), width = 15, height = 8) 
  plot(snd, type = 'l', xlab = 'Samples', ylab = 'Amplitude')
  dev.off() 
  
  spec = specgram(x = snd, n = nfft, Fs = fs, window = window, overlap = overlap)
  
  P = abs(spec$S)
  P = P/max(P)
  P = 10*log10(P)
  t = spec$t
  
  setwd("C:/Users/csia7/Desktop/OneDrive/OneDrive_2022-11-08/6CCM345A Third Year Project/R/R images/Spectrum(new)/Romance/Spectrogram (raw)")
  pdf(paste(gsub("c", "", rownames(finalDF)[i]), "pdf", sep=""), width = 15, height = 8) 
  image(P)
  dev.off() 
  
  setwd("C:/Users/csia7/Desktop/OneDrive/OneDrive_2022-11-08/6CCM345A Third Year Project/R/R images/Spectrum(new)/Romance/Spectrogram (processed)")
  pdf(paste(gsub("c", "", rownames(finalDF)[i]), "pdf", sep=""), width = 15, height = 8) 
  imagep(x = t, y = spec$f, z = t(P), col = oce.colorsViridis, ylab = 'Frequency [Hz]', xlab = 'Time [s]', drawPalette = T, decimate = F
  )
  dev.off()
  
  P <- c(t(P))
  specData <- rbind(specData, P)
}

finalSpec <- data.frame(specData)
finalSpec <- finalSpec[-1, ]

for (i in 1:nrow(finalSpec)){
  rownames(finalSpec)[i] <- names(Data)[i]
}

saveSpec <- finalSpec

rm(spec)
rm(specData)
\end{verbatim}

This is the code used for applying the PCA technique on interpolated dataframe and plotting relevant figures.
\begin{verbatim}
###################### PCA #####################


help(prcomp)
library(factoextra)

my_pca <- prcomp(finalDF, scale = TRUE, center = TRUE, retx = T)

names(my_pca);
summary(my_pca);

my_pca$sdev
my_pca$rotation
my_pca$center
my_pca$scale
my_pca$x
dim(my_pca$x);

setwd("C:/Users/csia7/Desktop/OneDrive/OneDrive_2022-11-08/6CCM345A Third Year Project/R/R images/PCA/rotations")
pdf("PC1 rotation.pdf", width = 15, height = 8) 
plot(my_pca$rotation[,1], type = "l", xlab = "PC1", ylab = "rotation")
dev.off()

pdf("PC2 rotation.pdf", width = 15, height = 8) 
plot(my_pca$rotation[,2], type = "l", xlab = "PC2", ylab = "rotation")
dev.off()

pdf("PC3 rotation.pdf", width = 15, height = 8) 
plot(my_pca$rotation[,3], type = "l", xlab = "PC3", ylab = "rotation")
dev.off()

setwd("C:/Users/csia7/Desktop/OneDrive/OneDrive_2022-11-08/6CCM345A Third Year Project/R/R images/PCA/PC biplots")
pdf("PC1 and PC2.pdf", width = 15, height = 8) 
plot(my_pca$x[,1], my_pca$x[,2], cex = 1, xlab = "PC1", ylab = "PC2", main = "PC1 and PC2")
dev.off()

pdf("PC1 and PC3.pdf", width = 15, height = 8) 
plot(my_pca$x[,1], my_pca$x[,3], cex = 1, xlab = "PC1", ylab = "PC3", main = "PC1 and PC3")
dev.off()

pdf("PC2 and PC3.pdf", width = 15, height = 8) 
plot(my_pca$x[,2], my_pca$x[,3], cex = 1, xlab = "PC2", ylab = "PC3", main = "PC2 and PC3")
dev.off()

# Compute variance

my_pca.var <- my_pca$sdev ^ 2;
my_pca.var;

# Proportion of variance for a scree plot #

propve <- my_pca.var / sum(my_pca.var);
propve;

setwd("C:/Users/csia7/Desktop/OneDrive/OneDrive_2022-11-08/6CCM345A Third Year Project/R/R images/PCA/Scree Plot")
pdf("Scree Plot.pdf", width = 15, height = 8) 
plot(propve, xlab = "Principal Component",
     ylab = "Proportion of Variance",
     ylim = c(0, 1), type = "b",
     main = "Scree Plot");
dev.off()

# Plot the cumulative proportion of variance explained

pdf("Cumulative Scree Plot.pdf", width = 15, height = 8) 
plot(cumsum(propve),
     xlab = "Principal Component",
     ylab = "Cumulative Proportion of Variance",
     ylim = c(0, 1), 
     type = "b",
     main = "Cumulative Scree Plot");
dev.off()

#Find Top n principal component
# which will at least cover 90 % variance of dimension

dim(my_pca$x) # 219
ncovdata <- which(cumsum(propve) >= 0.8)[1] 
ncovdata # 56/219
ncovdata <- which(cumsum(propve) >= 0.9)[1] 
ncovdata # 85/219
ncovdata <- which(cumsum(propve) >= 0.95)[1] 
ncovdata # 112/219

rm(my_pca)
\end{verbatim}

This is the code used for applying the PCA technique on spectrum dataframe and plotting relevant figures.
\begin{verbatim}
################### PCA on Spectrum ################


my_pca <- prcomp(finalSpec, scale = TRUE, center = TRUE, retx = T)

names(my_pca);
summary(my_pca);

my_pca$sdev
my_pca$rotation
my_pca$center
my_pca$scale
my_pca$x
dim(my_pca$x);

setwd("C:/Users/csia7/Desktop/OneDrive/OneDrive_2022-11-08/6CCM345A Third Year Project/R/R images/PCA Spectrum/rotations")
pdf("PC1 rotation.pdf", width = 15, height = 8) 
plot(my_pca$rotation[,1], type = "l", xlab = "PC1", ylab = "rotation")
dev.off()

pdf("PC2 rotation.pdf", width = 15, height = 8) 
plot(my_pca$rotation[,2], type = "l", xlab = "PC2", ylab = "rotation")
dev.off()

pdf("PC3 rotation.pdf", width = 15, height = 8) 
plot(my_pca$rotation[,3], type = "l", xlab = "PC3", ylab = "rotation")
dev.off()

setwd("C:/Users/csia7/Desktop/OneDrive/OneDrive_2022-11-08/6CCM345A Third Year Project/R/R images/PCA Spectrum/PC biplots")
pdf("PC1 and PC2.pdf", width = 15, height = 8) 
plot(my_pca$x[,1], my_pca$x[,2], cex = 1, xlab = "PC1", ylab = "PC2", main = "PC1 and PC2")
dev.off()

pdf("PC1 and PC3.pdf", width = 15, height = 8) 
plot(my_pca$x[,1], my_pca$x[,3], cex = 1, xlab = "PC1", ylab = "PC3", main = "PC1 and PC3")
dev.off()

pdf("PC2 and PC3.pdf", width = 15, height = 8) 
plot(my_pca$x[,2], my_pca$x[,3], cex = 1, xlab = "PC2", ylab = "PC3", main = "PC2 and PC3")
dev.off()

# Compute variance

my_pca.var <- my_pca$sdev ^ 2;
my_pca.var;

# Proportion of variance for a scree plot

propve <- my_pca.var / sum(my_pca.var);
propve;

setwd("C:/Users/csia7/Desktop/OneDrive/OneDrive_2022-11-08/6CCM345A Third Year Project/R/R images/PCA Spectrum/Scree Plot")
pdf("Scree Plot.pdf", width = 15, height = 8) 
plot(propve, xlab = "Principal Component",
     ylab = "Proportion of Variance",
     ylim = c(0, 1), type = "b",
     main = "Scree Plot");
dev.off()

# Plot the cumulative proportion of variance explained

pdf("Cumulative Scree Plot.pdf", width = 15, height = 8) 
plot(cumsum(propve),
     xlab = "Principal Component",
     ylab = "Cumulative Proportion of Variance",
     ylim = c(0, 1), 
     type = "b",
     main = "Cumulative Scree Plot");
dev.off()

# Find Top n principal component
# which will atleast cover 90 % variance of dimension

ncovdata <- which(cumsum(propve) >= 0.8)[1] 
ncovdata # 35
ncovdata <- which(cumsum(propve) >= 0.9)[1] 
ncovdata # 92
ncovdata <- which(cumsum(propve) >= 0.95)[1] 
ncovdata # 142
\end{verbatim}

This is the code used for applying the PCA technique on Portuguese section of the interpolated dataframe and plotting relevant figures.
\begin{verbatim}
######################## PCA (Portuguese) ########################


library(dplyr)

finalDFPO <- finalDF[1:25, ]
finalDFPOM <- finalDF[1:9, ]
finalDFPOF <- finalDF[10:25, ]

my_pca <- prcomp(finalDFPO, scale = TRUE, center = TRUE, retx = T)

names(my_pca);
summary(my_pca);

my_pca$sdev
my_pca$rotation
my_pca$center
my_pca$scale
my_pca$x
dim(my_pca$x);

setwd("C:/Users/csia7/Desktop/OneDrive/OneDrive_2022-11-08/6CCM345A Third Year Project/R/R images/PCA (Portuguese)/rotations")
pdf("PC1 rotation.pdf", width = 15, height = 8) 
plot(my_pca$rotation[,1], type = "l", xlab = "PC1", ylab = "rotation")
dev.off()

pdf("PC2 rotation.pdf", width = 15, height = 8) 
plot(my_pca$rotation[,2], type = "l", xlab = "PC2", ylab = "rotation")
dev.off()

pdf("PC3 rotation.pdf", width = 15, height = 8) 
plot(my_pca$rotation[,3], type = "l", xlab = "PC3", ylab = "rotation")
dev.off()

setwd("C:/Users/csia7/Desktop/OneDrive/OneDrive_2022-11-08/6CCM345A Third Year Project/R/R images/PCA (Portuguese)/PC biplots")
pdf("PC1 and PC2.pdf", width = 15, height = 8) 
plot(my_pca$x[,1], my_pca$x[,2], cex = 1, xlab = "PC1", ylab = "PC2", main = "PC1 and PC2")
dev.off()

pdf("PC1 and PC3.pdf", width = 15, height = 8) 
plot(my_pca$x[,1], my_pca$x[,3], cex = 1, xlab = "PC1", ylab = "PC3", main = "PC1 and PC3")
dev.off()

pdf("PC2 and PC3.pdf", width = 15, height = 8) 
plot(my_pca$x[,2], my_pca$x[,3], cex = 1, xlab = "PC2", ylab = "PC3", main = "PC2 and PC3")
dev.off()

# Compute variance

my_pca.var <- my_pca$sdev ^ 2;
my_pca.var;

# Proportion of variance for a scree plot

propve <- my_pca.var / sum(my_pca.var);
propve;

setwd("C:/Users/csia7/Desktop/OneDrive/OneDrive_2022-11-08/6CCM345A Third Year Project/R/R images/PCA (Portuguese)/Scree Plot")
pdf("Scree Plot.pdf", width = 15, height = 8) 
plot(propve, xlab = "Principal Component",
     ylab = "Proportion of Variance",
     ylim = c(0, 1), type = "b",
     main = "Scree Plot");
dev.off()

# Plot the cumulative proportion of variance explained

pdf("Cumulative Scree Plot.pdf", width = 15, height = 8) 
plot(cumsum(propve),
     xlab = "Principal Component",
     ylab = "Cumulative Proportion of Variance",
     ylim = c(0, 1), 
     type = "b",
     main = "Cumulative Scree Plot");
dev.off()

# Find Top n principal component
# which will at least cover 90 % variance of dimension

dim(my_pca$x) # 25
ncovdata <- which(cumsum(propve) >= 0.8)[1] 
ncovdata # 14/25
ncovdata <- which(cumsum(propve) >= 0.9)[1] 
ncovdata # 17/25
ncovdata <- which(cumsum(propve) >= 0.95)[1] 
ncovdata # 20/25
\end{verbatim}

This is the code used for applying the PCA technique on Italian section of the interpolated dataframe and plotting relevant figures.
\begin{verbatim}
######################## PCA (Italian) ########################

library(dplyr)

finalDFIT <- finalDF[26:75, ]

my_pca <- prcomp(finalDFIT, scale = TRUE, center = TRUE, retx = T)

names(my_pca);
summary(my_pca);

my_pca$sdev
my_pca$rotation
my_pca$center
my_pca$scale
my_pca$x
dim(my_pca$x);

setwd("C:/Users/csia7/Desktop/OneDrive/OneDrive_2022-11-08/6CCM345A Third Year Project/R/R images/PCA (Italian)/rotations")
pdf("PC1 rotation.pdf", width = 15, height = 8) 
plot(my_pca$rotation[,1], type = "l", xlab = "PC1", ylab = "rotation")
dev.off()

pdf("PC2 rotation.pdf", width = 15, height = 8) 
plot(my_pca$rotation[,2], type = "l", xlab = "PC2", ylab = "rotation")
dev.off()

pdf("PC3 rotation.pdf", width = 15, height = 8) 
plot(my_pca$rotation[,3], type = "l", xlab = "PC3", ylab = "rotation")
dev.off()

setwd("C:/Users/csia7/Desktop/OneDrive/OneDrive_2022-11-08/6CCM345A Third Year Project/R/R images/PCA (Italian)/PC biplots")
pdf("PC1 and PC2.pdf", width = 15, height = 8) 
plot(my_pca$x[,1], my_pca$x[,2], cex = 1, xlab = "PC1", ylab = "PC2", main = "PC1 and PC2")
dev.off()

pdf("PC1 and PC3.pdf", width = 15, height = 8) 
plot(my_pca$x[,1], my_pca$x[,3], cex = 1, xlab = "PC1", ylab = "PC3", main = "PC1 and PC3")
dev.off()

pdf("PC2 and PC3.pdf", width = 15, height = 8) 
plot(my_pca$x[,2], my_pca$x[,3], cex = 1, xlab = "PC2", ylab = "PC3", main = "PC2 and PC3")
dev.off()

# Compute variance

my_pca.var <- my_pca$sdev ^ 2;
my_pca.var;

# Proportion of variance for a scree plot

propve <- my_pca.var / sum(my_pca.var);
propve;

setwd("C:/Users/csia7/Desktop/OneDrive/OneDrive_2022-11-08/6CCM345A Third Year Project/R/R images/PCA (Italian)/Scree Plot")
pdf("Scree Plot.pdf", width = 15, height = 8) 
plot(propve, xlab = "Principal Component",
     ylab = "Proportion of Variance",
     ylim = c(0, 1), type = "b",
     main = "Scree Plot");
dev.off()

# Plot the cumulative proportion of variance explained

pdf("Cumulative Scree Plot.pdf", width = 15, height = 8) 
plot(cumsum(propve),
     xlab = "Principal Component",
     ylab = "Cumulative Proportion of Variance",
     ylim = c(0, 1), 
     type = "b",
     main = "Cumulative Scree Plot");
dev.off()

# Find Top n principal component
# which will at least cover 90 % variance of dimension

dim(my_pca$x) # 50
ncovdata <- which(cumsum(propve) >= 0.8)[1] 
ncovdata # 20/50
ncovdata <- which(cumsum(propve) >= 0.9)[1] 
ncovdata # 29/50
ncovdata <- which(cumsum(propve) >= 0.95)[1] 
ncovdata # 35/50
\end{verbatim}

This is the code used for applying the PCA technique on Spanish Italian section of the interpolated dataframe and plotting relevant figures.
\begin{verbatim}
##################### PCA (Spanish Italian) #####################


library(dplyr)

finalDFSI <- finalDF[76:113, ]
finalDFSIM <- finalDF[85:94, ]
finalDFSIM <- bind_rows(finalDFSIM, finalDF[104:113, ])
finalDFSIF <- finalDF[76:84, ]
finalDFSIF <- bind_rows(finalDFSIF, finalDF[95:103, ])

dim(finalDFSI)
dim(finalDFSIM)
dim(finalDFSIF)

my_pca <- prcomp(finalDFSI, scale = TRUE, center = TRUE, retx = T)

names(my_pca);
summary(my_pca);

my_pca$sdev
my_pca$rotation
my_pca$center
my_pca$scale
my_pca$x
dim(my_pca$x);

setwd("C:/Users/csia7/Desktop/OneDrive/OneDrive_2022-11-08/6CCM345A Third Year Project/R/R images/PCA (Iberian Spanish)/rotations")
pdf("PC1 rotation.pdf", width = 15, height = 8) 
plot(my_pca$rotation[,1], type = "l", xlab = "PC1", ylab = "rotation")
dev.off()

pdf("PC2 rotation.pdf", width = 15, height = 8) 
plot(my_pca$rotation[,2], type = "l", xlab = "PC2", ylab = "rotation")
dev.off()

pdf("PC3 rotation.pdf", width = 15, height = 8) 
plot(my_pca$rotation[,3], type = "l", xlab = "PC3", ylab = "rotation")
dev.off()

setwd("C:/Users/csia7/Desktop/OneDrive/OneDrive_2022-11-08/6CCM345A Third Year Project/R/R images/PCA (Iberian Spanish)/PC biplots")
pdf("PC1 and PC2.pdf", width = 15, height = 8) 
plot(my_pca$x[,1], my_pca$x[,2], cex = 1, xlab = "PC1", ylab = "PC2", main = "PC1 and PC2")
dev.off()

pdf("PC1 and PC3.pdf", width = 15, height = 8) 
plot(my_pca$x[,1], my_pca$x[,3], cex = 1, xlab = "PC1", ylab = "PC3", main = "PC1 and PC3")
dev.off()

pdf("PC2 and PC3.pdf", width = 15, height = 8) 
plot(my_pca$x[,2], my_pca$x[,3], cex = 1, xlab = "PC2", ylab = "PC3", main = "PC2 and PC3")
dev.off()

# Compute variance

my_pca.var <- my_pca$sdev ^ 2;
my_pca.var;

# Proportion of variance for a scree plot

propve <- my_pca.var / sum(my_pca.var);
propve;

setwd("C:/Users/csia7/Desktop/OneDrive/OneDrive_2022-11-08/6CCM345A Third Year Project/R/R images/PCA (Iberian Spanish)/Scree Plot")
pdf("Scree Plot.pdf", width = 15, height = 8) 
plot(propve, xlab = "Principal Component",
     ylab = "Proportion of Variance",
     ylim = c(0, 1), type = "b",
     main = "Scree Plot");
dev.off()

# Plot the cumulative proportion of variance explained

pdf("Cumulative Scree Plot.pdf", width = 15, height = 8) 
plot(cumsum(propve),
     xlab = "Principal Component",
     ylab = "Cumulative Proportion of Variance",
     ylim = c(0, 1), 
     type = "b",
     main = "Cumulative Scree Plot");
dev.off()

# Find Top n principal component
# which will at least cover 90 % variance of dimension

dim(my_pca$x) # 38
ncovdata <- which(cumsum(propve) >= 0.8)[1] 
ncovdata # 15/38
ncovdata <- which(cumsum(propve) >= 0.9)[1] 
ncovdata # 21/38
ncovdata <- which(cumsum(propve) >= 0.95)[1] 
ncovdata # 26/38
\end{verbatim}

This is the code used for applying the PCA technique on French section of the interpolated dataframe and plotting relevant figures.
\begin{verbatim}
######################## PCA (French) ########################


library(dplyr)

finalDFFR <- finalDF[114:173, ]
finalDFFRM <- finalDF[134:143, ]
finalDFFRM <- bind_rows(finalDFFRM, finalDF[160:163, ])
finalDFFRF <- finalDF[114:133, ]
finalDFFRF <- bind_rows(finalDFFRF, finalDF[144:159, ])
finalDFFRF <- bind_rows(finalDFFRF, finalDF[164:173, ])

dim(finalDFFR)
dim(finalDFFRM)
dim(finalDFFRF)

my_pca <- prcomp(finalDFFR, scale = TRUE, center = TRUE, retx = T)

names(my_pca);
summary(my_pca);

my_pca$sdev
my_pca$rotation
my_pca$center
my_pca$scale
my_pca$x
dim(my_pca$x);

setwd("C:/Users/csia7/Desktop/OneDrive/OneDrive_2022-11-08/6CCM345A Third Year Project/R/R images/PCA (French)/rotations")
pdf("PC1 rotation.pdf", width = 15, height = 8) 
plot(my_pca$rotation[,1], type = "l", xlab = "PC1", ylab = "rotation")
dev.off()

pdf("PC2 rotation.pdf", width = 15, height = 8) 
plot(my_pca$rotation[,2], type = "l", xlab = "PC2", ylab = "rotation")
dev.off()

pdf("PC3 rotation.pdf", width = 15, height = 8) 
plot(my_pca$rotation[,3], type = "l", xlab = "PC3", ylab = "rotation")
dev.off()

setwd("C:/Users/csia7/Desktop/OneDrive/OneDrive_2022-11-08/6CCM345A Third Year Project/R/R images/PCA (French)/PC biplots")
pdf("PC1 and PC2.pdf", width = 15, height = 8) 
plot(my_pca$x[,1], my_pca$x[,2], cex = 1, xlab = "PC1", ylab = "PC2", main = "PC1 and PC2")
dev.off()

pdf("PC1 and PC3.pdf", width = 15, height = 8) 
plot(my_pca$x[,1], my_pca$x[,3], cex = 1, xlab = "PC1", ylab = "PC3", main = "PC1 and PC3")
dev.off()

pdf("PC2 and PC3.pdf", width = 15, height = 8) 
plot(my_pca$x[,2], my_pca$x[,3], cex = 1, xlab = "PC2", ylab = "PC3", main = "PC2 and PC3")
dev.off()

# Compute variance

my_pca.var <- my_pca$sdev ^ 2;
my_pca.var;

# Proportion of variance for a scree plot

propve <- my_pca.var / sum(my_pca.var);
propve;

setwd("C:/Users/csia7/Desktop/OneDrive/OneDrive_2022-11-08/6CCM345A Third Year Project/R/R images/PCA (French)/Scree Plot")
pdf("Scree Plot.pdf", width = 15, height = 8) 
plot(propve, xlab = "Principal Component",
     ylab = "Proportion of Variance",
     ylim = c(0, 1), type = "b",
     main = "Scree Plot");
dev.off()

# Plot the cumulative proportion of variance explained

pdf("Cumulative Scree Plot.pdf", width = 15, height = 8) 
plot(cumsum(propve),
     xlab = "Principal Component",
     ylab = "Cumulative Proportion of Variance",
     ylim = c(0, 1), 
     type = "b",
     main = "Cumulative Scree Plot");
dev.off()

# Find Top n principal component
# which will at least cover 90 % variance of dimension

dim(my_pca$x) # 60
ncovdata <- which(cumsum(propve) >= 0.8)[1] 
ncovdata # 22/60
ncovdata <- which(cumsum(propve) >= 0.9)[1] 
ncovdata # 31/60
ncovdata <- which(cumsum(propve) >= 0.95)[1] 
ncovdata # 39/60
\end{verbatim}

This is the code used for applying the PCA technique on American Spanish section of the interpolated dataframe and plotting relevant figures.
\begin{verbatim}
##################### PCA (American Spanish) #####################

library(dplyr)

finalDFSA <- finalDF[174:219, ]
finalDFSAM <- finalDF[180:189, ]
finalDFSAF <- finalDF[174:179, ]
finalDFSAF <- bind_rows(finalDFSAF, finalDF[190:219, ])

dim(finalDFSA)
dim(finalDFSAM)
dim(finalDFSAF)

my_pca <- prcomp(finalDFSA, scale = TRUE, center = TRUE, retx = T)

names(my_pca);
summary(my_pca);

my_pca$sdev
my_pca$rotation
my_pca$center
my_pca$scale
my_pca$x
dim(my_pca$x);

setwd("C:/Users/csia7/Desktop/OneDrive/OneDrive_2022-11-08/6CCM345A Third Year Project/R/R images/PCA (American Spanish)/rotations")
pdf("PC1 rotation.pdf", width = 15, height = 8) 
plot(my_pca$rotation[,1], type = "l", xlab = "PC1", ylab = "rotation")
dev.off()

pdf("PC2 rotation.pdf", width = 15, height = 8) 
plot(my_pca$rotation[,2], type = "l", xlab = "PC2", ylab = "rotation")
dev.off()

pdf("PC3 rotation.pdf", width = 15, height = 8) 
plot(my_pca$rotation[,3], type = "l", xlab = "PC3", ylab = "rotation")
dev.off()

setwd("C:/Users/csia7/Desktop/OneDrive/OneDrive_2022-11-08/6CCM345A Third Year Project/R/R images/PCA (American Spanish)/PC biplots")
pdf("PC1 and PC2.pdf", width = 15, height = 8) 
plot(my_pca$x[,1], my_pca$x[,2], cex = 1, xlab = "PC1", ylab = "PC2", main = "PC1 and PC2")
dev.off()

pdf("PC1 and PC3.pdf", width = 15, height = 8) 
plot(my_pca$x[,1], my_pca$x[,3], cex = 1, xlab = "PC1", ylab = "PC3", main = "PC1 and PC3")
dev.off()

pdf("PC2 and PC3.pdf", width = 15, height = 8) 
plot(my_pca$x[,2], my_pca$x[,3], cex = 1, xlab = "PC2", ylab = "PC3", main = "PC2 and PC3")
dev.off()

# Compute variance

my_pca.var <- my_pca$sdev ^ 2;
my_pca.var;

# Proportion of variance for a scree plot

propve <- my_pca.var / sum(my_pca.var);
propve;

setwd("C:/Users/csia7/Desktop/OneDrive/OneDrive_2022-11-08/6CCM345A Third Year Project/R/R images/PCA (American Spanish)/Scree Plot")
pdf("Scree Plot.pdf", width = 15, height = 8) 
plot(propve, xlab = "Principal Component",
     ylab = "Proportion of Variance",
     ylim = c(0, 1), type = "b",
     main = "Scree Plot");
dev.off()

# Plot the cumulative proportion of variance explained

pdf("Cumulative Scree Plot.pdf", width = 15, height = 8) 
plot(cumsum(propve),
     xlab = "Principal Component",
     ylab = "Cumulative Proportion of Variance",
     ylim = c(0, 1), 
     type = "b",
     main = "Cumulative Scree Plot");
dev.off()

# Find Top n principal component
# which will at least cover 90 % variance of dimension

dim(my_pca$x) # 46
ncovdata <- which(cumsum(propve) >= 0.8)[1] 
ncovdata # 7/46
ncovdata <- which(cumsum(propve) >= 0.9)[1] 
ncovdata # 11/46
ncovdata <- which(cumsum(propve) >= 0.95)[1] 
ncovdata # 18/46
\end{verbatim}

This is the code used for applying the ICA technique on the interpolated dataframe and plotting relevant figures.
\begin{verbatim}
###################### ICA ####################


install.packages("fastICA")
install.packages("scatterplot3d")
library(fastICA)
library(scatterplot3d)

# ICA with n.comp=5 

ica5 <- fastICA(finalDF, n.comp=5)

setwd("C:/Users/csia7/Desktop/OneDrive/OneDrive_2022-11-08/6CCM345A Third Year Project/R/R images/ICA/[5] 5x5 plot")

pairs(ica5$S, col=rainbow(219))

setwd("C:/Users/csia7/Desktop/OneDrive/OneDrive_2022-11-08/6CCM345A Third Year Project/R/R images/ICA/[5] individual plots")

for (i in 1:5){
  pdf(paste(paste("comp", i, sep=" "), ".pdf", sep=""), width = 5, height = 5) 
  plot(ica5$S[,i],
       ica5$S[,i], 
       col=rainbow(219), 
       xlab=paste("Comp", i, sep=" "), 
       ylab=paste("Comp", i, sep=" "))
  dev.off()
}

uniVal5 <- list()
for (i in 1:5){
  Comp <- sort(ica5$S[,i])
  uniComp <- seq(min(ica5$S[,i]), max(ica5$S[,i]), length.out = 219)
  uniVal <- 0
  for (j in 1:219){
    closeVal <- 0
    closeVal <- which(abs(Comp-uniComp[j])==min(abs(Comp-uniComp[j])))
    uniVal <- uniVal + abs(Comp[closeVal]-uniComp[j])
  }
  uniVal5 <- append(uniVal5, uniVal)
}
fitComp <- which.min(uniVal5)
fitComp

setwd("C:/Users/csia7/Desktop/OneDrive/OneDrive_2022-11-08/6CCM345A Third Year Project/R/R images/ICA/[5] individual plots/fittest")

pdf(paste(paste("comp", fitComp, sep=" "), ".pdf", sep=""), width = 5, height = 5) 
plot(ica5$S[,fitComp],
     ica5$S[,fitComp], 
     col=rainbow(219), 
     xlab=paste("Comp", fitComp, sep=" "), 
     ylab=paste("Comp", fitComp, sep=" "))
dev.off()

#ICA with n.comp=10 

ica10 <- fastICA(finalDF, n.comp=10)

setwd("C:/Users/csia7/Desktop/OneDrive/OneDrive_2022-11-08/6CCM345A Third Year Project/R/R images/ICA/[10] 10x10 plot")

pairs(ica10$S, col=rainbow(10))

setwd("C:/Users/csia7/Desktop/OneDrive/OneDrive_2022-11-08/6CCM345A Third Year Project/R/R images/ICA/[10] 5x5 plots")

pairs(ica10$S[,1:5], col=rainbow(10))
pairs(ica10$S[,6:10], col=rainbow(10))

setwd("C:/Users/csia7/Desktop/OneDrive/OneDrive_2022-11-08/6CCM345A Third Year Project/R/R images/ICA/[10] individual plots")

for (i in 1:10){
  pdf(paste(paste("comp", i, sep=" "), ".pdf", sep=""), width = 5, height = 5) 
  plot(ica10$S[,i],
       ica10$S[,i], 
       col=rainbow(219), 
       xlab=paste("Comp", i, sep=" "), 
       ylab=paste("Comp", i, sep=" "))
  dev.off()
}

uniVal10new <- list()
for (i in 1:10){
  Comp <- sort(ica10$S[,i])
  uniComp <- seq(min(ica10$S[,i]), max(ica10$S[,i]), length.out = 219)
  uniVal <- 0
  for (j in 1:219){
    closeVal <- 0
    closeVal <- which(abs(Comp-uniComp[j])==min(abs(Comp-uniComp[j])))
    uniVal <- uniVal + abs(Comp[closeVal]-uniComp[j])
  }
  uniVal10new <- append(uniVal10new, uniVal)
}
fitComp <- which.min(uniVal10new)
fitComp

setwd("C:/Users/csia7/Desktop/OneDrive/OneDrive_2022-11-08/6CCM345A Third Year Project/R/R images/ICA/[10] individual plots/fittest")

pdf(paste(paste("comp", fitComp, sep=" "), ".pdf", sep=""), width = 5, height = 5) 
plot(ica10$S[,fitComp],
     ica10$S[,fitComp], 
     col=rainbow(219), 
     xlab=paste("Comp", fitComp, sep=" "), 
     ylab=paste("Comp", fitComp, sep=" "))
dev.off()

# ICA with n.comp=20

ica20 <- fastICA(finalDF, n.comp=20)

setwd("C:/Users/csia7/Desktop/OneDrive/OneDrive_2022-11-08/6CCM345A Third Year Project/R/R images/ICA/[20] 20x20 plot")

pairs(ica20$S, col=rainbow(219))

setwd("C:/Users/csia7/Desktop/OneDrive/OneDrive_2022-11-08/6CCM345A Third Year Project/R/R images/ICA/[20] 5x5 plots")

pairs(ica20$S[,1:5], col=rainbow(219))
pairs(ica20$S[,6:10], col=rainbow(219))
pairs(ica20$S[,11:15], col=rainbow(219))
pairs(ica20$S[,16:20], col=rainbow(219))

setwd("C:/Users/csia7/Desktop/OneDrive/OneDrive_2022-11-08/6CCM345A Third Year Project/R/R images/ICA/[20] individual plots")

for (i in 1:20){
  pdf(paste(paste("comp", i, sep=" "), ".pdf", sep=""), width = 5, height = 5) 
  plot(ica20$S[,i],
       ica20$S[,i], 
       col=rainbow(219), 
       xlab=paste("Comp", i, sep=" "), 
       ylab=paste("Comp", i, sep=" "))
  dev.off()
}

uniVal20new <- list()
for (i in 1:20){
  Comp <- sort(ica20$S[,i])
  uniComp <- seq(min(ica20$S[,i]), max(ica20$S[,i]), length.out = 219)
  uniVal <- 0
  for (j in 1:219){
    closeVal <- 0
    closeVal <- which(abs(Comp-uniComp[j])==min(abs(Comp-uniComp[j])))
    uniVal <- uniVal + abs(Comp[closeVal]-uniComp[j])
  }
  uniVal20new <- append(uniVal20new, uniVal)
}
fitComp <- which.min(uniVal20new)
fitComp

setwd("C:/Users/csia7/Desktop/OneDrive/OneDrive_2022-11-08/6CCM345A Third Year Project/R/R images/ICA/[20] individual plots/fittest")

pdf(paste(paste("comp", fitComp, sep=" "), ".pdf", sep=""), width = 5, height = 5) 
plot(ica20$S[,fitComp],
     ica20$S[,fitComp], 
     col=rainbow(219), 
     xlab=paste("Comp", fitComp, sep=" "), 
     ylab=paste("Comp", fitComp, sep=" "))
dev.off()

#ICA with n.comp=40 

ica40 <- fastICA(finalDF, n.comp=40)

setwd("C:/Users/csia7/Desktop/OneDrive/OneDrive_2022-11-08/6CCM345A Third Year Project/R/R images/ICA/[40] 5x5 plots")

pairs(ica40$S[,1:5], col=rainbow(10))
pairs(ica40$S[,6:10], col=rainbow(10))
pairs(ica40$S[,11:15], col=rainbow(10))
pairs(ica40$S[,16:20], col=rainbow(10))
pairs(ica40$S[,21:25], col=rainbow(10))
pairs(ica40$S[,26:30], col=rainbow(10))
pairs(ica40$S[,31:35], col=rainbow(10))
pairs(ica40$S[,36:40], col=rainbow(10))

setwd("C:/Users/csia7/Desktop/OneDrive/OneDrive_2022-11-08/6CCM345A Third Year Project/R/R images/ICA/[40] individual plots")

for (i in 1:40){
  pdf(paste(paste("comp", i, sep=" "), ".pdf", sep=""), width = 5, height = 5) 
  plot(ica40$S[,i],
       ica40$S[,i], 
       col=rainbow(10), 
       xlab=paste("Comp", i, sep=" "), 
       ylab=paste("Comp", i, sep=" "))
  dev.off()
}

uniVal40new <- list()
for (i in 1:40){
  Comp <- sort(ica40$S[,i])
  uniComp <- seq(min(ica40$S[,i]), max(ica40$S[,i]), length.out = 219)
  uniVal <- 0
  for (j in 1:219){
    closeVal <- 0
    closeVal <- which(abs(Comp-uniComp[j])==min(abs(Comp-uniComp[j])))
    uniVal <- uniVal + abs(Comp[closeVal]-uniComp[j])
  }
  uniVal40new <- append(uniVal40new, uniVal)
}
fitComp <- which.min(uniVal40new)
fitComp

setwd("C:/Users/csia7/Desktop/OneDrive/OneDrive_2022-11-08/6CCM345A Third Year Project/R/R images/ICA/[40] individual plots/fittest")

pdf(paste(paste("comp", fitComp, sep=" "), ".pdf", sep=""), width = 5, height = 5) 
plot(ica40$S[,fitComp],
     ica40$S[,fitComp], 
     col=rainbow(219), 
     xlab=paste("Comp", fitComp, sep=" "), 
     ylab=paste("Comp", fitComp, sep=" "))
dev.off()

# ICA with n.comp=80 

ica80 <- fastICA(finalDF, n.comp=80)

setwd("C:/Users/csia7/Desktop/OneDrive/OneDrive_2022-11-08/6CCM345A Third Year Project/R/R images/ICA/[80] 5x5 plots")

pairs(ica80$S[,1:5], col=rainbow(10))
pairs(ica80$S[,6:10], col=rainbow(10))
pairs(ica80$S[,11:15], col=rainbow(10))
pairs(ica80$S[,16:20], col=rainbow(10))

setwd("C:/Users/csia7/Desktop/OneDrive/OneDrive_2022-11-08/6CCM345A Third Year Project/R/R images/ICA/[80] individual plots")

for (i in 1:80){
  pdf(paste(paste("comp", i, sep=" "), ".pdf", sep=""), width = 5, height = 5) 
  plot(ica80$S[,i],
       ica80$S[,i], 
       col=rainbow(10), 
       xlab=paste("Comp", i, sep=" "), 
       ylab=paste("Comp", i, sep=" "))
  dev.off()
}

uniVal80new <- list()
for (i in 1:80){
  Comp <- sort(ica80$S[,i])
  uniComp <- seq(min(ica80$S[,i]), max(ica80$S[,i]), length.out = 219)
  uniVal <- 0
  for (j in 1:219){
    closeVal <- 0
    closeVal <- which(abs(Comp-uniComp[j])==min(abs(Comp-uniComp[j])))
    uniVal <- uniVal + abs(Comp[closeVal]-uniComp[j])
  }
  uniVal80new <- append(uniVal80new, uniVal)
}
fitComp <- which.min(uniVal80new)
fitComp

setwd("C:/Users/csia7/Desktop/OneDrive/OneDrive_2022-11-08/6CCM345A Third Year Project/R/R images/ICA/[80] individual plots/fittest")

pdf(paste(paste("comp", fitComp, sep=" "), ".pdf", sep=""), width = 5, height = 5) 
plot(ica80$S[,fitComp],
     ica80$S[,fitComp], 
     col=rainbow(219), 
     xlab=paste("Comp", fitComp, sep=" "), 
     ylab=paste("Comp", fitComp, sep=" "))
dev.off()

uniVal80new[50]
\end{verbatim}